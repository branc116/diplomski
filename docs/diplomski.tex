\documentclass[times, utf8, diplomski]{fer}
\usepackage{booktabs}

\begin{document}

\thesisnumber{3155}

\title{Programska podrška za pouzdano prikupljanje podataka s ugradbenih sustava}

\author{Branimir Ričko}

\maketitle

\zahvala{Zahvaljujem se Lani na tome \v{s}to mi je napisala cijeli diplomski. <3}

\tableofcontents

\chapter{Uvod}
Ugradbena računala nude brojne mogućnosti poboljšanja svakodnevice pojedinaca. Njihove fizičke dimenzije i niska potrošnja cine ih dobrim kandidatom za stvaranjem uređaja i strojeva koji su pametniji on onih proizvedenih u prošlom desetljeću. Iako je bolja budućnost veoma blizu svima nama, iako postoje ugradbena računala koja nam omogućuju izgradnju pametnijih stvari, proces izgradnje nije nimalo trivijalan. Ovaj se rad bavi navedenim procesom i analizom pouzdanosti istog. Opisan je postupak izrade jednostavnog sustava s ugradbenim računalom gdje je ugradbeno računalo stm32, koje čita podatke sa senzora i iste u realnom vremenu \v{s}alje na server.

\chapter{Ugradbena računala}
Ugradbena računala su računala malih dimenzija i niske potro\v{s}nje. 

\section{STM32}
Stm32 je ugradbeno računalo. U ovom radu stm32 koristi se kao primjer računala na kojem je moguće razvijati pouzdan sustav.

\subsection{Sensortile}
Sensortile je integrirana plocica na kojoj se nalazi STM32L4 mikrokontroler i nekoliko senzora.

\subsubsection{LSM6DSM}
LSM6DSM je integrirani system-in-package koji sadr\v{z}i 3D digaitalni akcelorometar i gyroscope. Mo\v{z} batchati podatke ili ih slati u realnom
vremenu. Skala akcelerometra mo\v{z}e biti pode\v{s}ena na ±2/±4/±8/±16g. Kru\v{z}na brzina mo\v{z} biti pode\v{s}ena na
±125/±245/±500/±1000/±2000 dps.

\subsubsection{LSM303AGR}
LSM303AGR je integrirani system-in-package koji sadr\v{z}i 3D digaitalni akcelorometar i 3D magnetski senzor. Skala akcelerometra je podesiva na
±2/±4/±8/±16g. Magnetski senzor ima dinami\v{c}ki raspon do ±50 gausa. Koristi I\(^2\)C ili SPI su\v{c}elje za komunikaciju.

\subsubsection{LPS22HB}
LPS22HB je integrirani system-in-package koji sadr\v{z}i digitalni barometar. Koristi I\(^2\)C ili SPI su\v{c}elje za komunikaciju.

\subsubsection{MP34DT05-A}
MP34DT05-A je integrirani system-in-package koji sadr\v{z}i digitalni mikrofon. Koristi IC su\v{c}elje.

\subsubsection{BLUENRG-MS}
BLUENRG-MS je integrirani system-in-package koji sadr\v{z}i Cortex-M0 ARM jezgru i mogu\v{c}nost bluetooth 4.1 komunikacije.
Cijeli paket tro\v{s}i maksimalno 1mA za 1dBm izlazne snage. Za komunikaciju s glavnim mikrokontrolerom kositi se SPI komunikacija.

\section{Razvojno okru\v{z}enje}
Razvojno okru\v{z}enje je okru\v{z}enje u kojem se razvija.

\subsection{cmake}
Cmake je programski alat koji je koristan za definiranje koraka potrebnih za izgradnju projekta.

\subsection{GNU make}
GNU make je programski alat za definiranje \textit{recepta} i kako izgraditi pojedini element sustava.

\section{Upravljanje interuptovima}
Cortex m procesori posjeduju specifican nacim upravljanjem iznimkam. Iznimke su veoma bitne ko stvaranja ugradbenih sustva zato \v{s}to omugucuju energetski efikasan dizajn sustava.
Umijesto aktovnog propitkivanja od strane procesora \textit{nalazi li se \v{s}to na nekoj sabirnici}, procesor mo\v{z}e biti na hrdverskoj razini obavje\v{s}ten da je neka sabirnica
promjenila stanje.
Kad se dogodi neki interupt izvodi se sljedeca procedura:

\subsubsection{Suspend main program execution}
\begin{itemize}
  \item finish current instruction
  \item save CPU state (push registers onto stack)
  \item set LR to \textbf{0xFFFFFFF9} (indicates interrupt return)
  \item set IPSR to interrupt number
  \item load PC with ISR address from vector table
\end{itemize}

\subsubsection{Execute interrupt service routine (ISR)}
\begin{itemize}
  \item save other registers to be used
  \item clear the “flag” that requested the interrupt
  \item perform the requested service
  \item communicate with other routines via global variables
  \item restore any registers saved by the ISR
\end{itemize}

\subsubsection{xPSR registar}

\begin{table}[h]
  \begin{center}
    \begin{tabular}{c|c|c|c|c||c||c||c|}
      \cline{2-8} & 31 & 30 & 29 & 28 & 24 & 9 & 5-0\\
      \hline
      \multicolumn{1}{|c|}{APSR} & N & Z & C & V & \multicolumn{3}{c|}{ } \\
      \hline
      \multicolumn{1}{|c|}{IPSR} & \multicolumn{6}{c||}{ } & Broj iznimke \\
      \hline
      \multicolumn{1}{|c|}{EPSR} & \multicolumn{4}{c||}{ } & T & a & \\
      \hline
    \end{tabular}
    \caption{Sadr\v{z}aj xPSR registra}
  \end{center}
\end{table}

\newpage
\section{Bluetooth}
\subsection{BlueNRG}
BlueNRG je bluetooth module koji se nalazi na Sensortile plo\v{c}ici. Komunicira s mikrokontrolerom pomocu SPI protokola.
Specificno postoji
\begin{itemize}
  \item Zaglavlje
  \item Tijelo
\end{itemize}
Zaglavlje govori koliko bajta je BlueNRG spreman primiti te koliko je bajta \v{z}eljan poslati.
Postoji pet signala koji povezuju bluetooth module i STM32 mikrokontroler:

\begin{itemize}
  \item Odabir \v{c}ipa (engl. Chip select CS) Mikrokontroler stavlja na logi\v{c}ku nulu da ozna\v{c}i da BlueNRG smije komunicirati.
  \item Takt (engl. clock) Mikrokontroler odrduje takt kojim ce komunicirati
  \item STM32 govori, BlueNRG slu\v{s}a
  \item BlueNRG govori, STM32 slu\v{s}a
  \item Podaci su spremni (engl. Data ready) BlueNRG ozna\v{c}ava da su dostupni podaci koje STM32 mo\v{z}e pro\v{c}itati.
\end{itemize}

\subsection{Performase}
Kori\v{s}tenjem jednog znakovnog kanala (engl. Character stream) u jednoj sekundi moguce je poslati do 140 poruka od 20B.

\begin{table}[h]
  \begin{center}
    \begin{tabular}[c]{l|l|l|l}
      \hline
      \multicolumn{1}{c|}{\textbf{Broj kanala}} &
      \multicolumn{1}{c}{\textbf{Veli\v{c}ina poruke (bajt)}} &
      \multicolumn{1}{c}{\textbf{Poruka/sekunda}} &
      \multicolumn{1}{c}{\textbf{Bajt/sekunda}} \\
      \hline
      1 & 16 & 142 & 2272 \\
      1 & 20 & 144 & 2880 \\
      2 & 20 & 162 & 3240 \\
      \hline
    \end{tabular}
  \caption{Bluetooth performanse}
  \end{center}
\end{table}


\subsection{Bluetooth GAP}
Bluetooth \textbf{G}eneric \textbf{A}ccess \textbf{P}rofile.
Gap definiral role unutar bluetooth standard. Role koje postoje su Advertizer, Scanner, Slave, Master. Adertizer i Scanner role vr\v{s}e razmjenu informacija isklju\v{c}ivo u smjeru
od Advertizera do Scannera. Isto tako, nije potrebno eksplicitno spajati uredaje.
Vi\v{s}e Scannera mo\v{z}e slu\v{s}ati jednog Adverizera.

\subsection{Bluetooth na linux operativnom sustavu}

\begin{table}[h]
  \begin{center}
    \begin{tabular}[c]{l|r}
      \multicolumn{1}{c|}{\textbf{Paket}} & 
      \multicolumn{1}{c}{\textbf{Opis paketa}} \\
      \hline
      bluez-utils & Pomocni programi za debugiranje bluetooth protokola \\
      bluez-libs & Biblioteke implementacijom raznih bluetooth protokola \\
      bluez & Servis za automatsko upravljanje bluetooth adapterima \\
      \hline
    \end{tabular}
  \caption{Lista paketa potrebnih za ispravan rad bluetootha na linux os-u}
  \end{center}
\end{table}


\begin{equation}
  \pi=3
\end{equation}

\chapter{Zaključak}
Ugradbena računala su dobra.

\bibliography{literatura}
\bibliographystyle{fer}

\begin{sazetak}
  Ugradbena ra\v{c}unala nude brojne mogućnosti poboljšanja svakodnevice pojedinaca. Njihove fizičke dimenzije i niska potrošnja čine ih dobrim kandidatom za stvaranjem uređaja i strojeva koji su pametniji on onih proizvedenih u prošlom desetljeću. Iako je bolja budućnost veoma blizu svima nama, iako postoje ugradbena računala koja nam omogućuju izgradnju pametnijih stvari, proces izgradnje nije nimalo trivijalan. Ovaj se rad bavi navedenim procesom i analizom pouzdanosti istog.

\kljucnerijeci{Ugradbena računala, Pouzdanost, Bluetooth, STM32}
\end{sazetak}

\engtitle{Software support for data collection from embedded devices with high availability}

\begin{abstract}
Embedded computers offer number of features that can augment day to day exisstance of every individual. Their dimensions and power efficiency make them great candidate for making a world a better place. Even thou embedded devices exist and are availabel to everyone, creating a system that uses that device is not trivial. This work touches on those processies and analasys of availability of those processies.

\keywords{Embedded Computers, Availability, Bluetooth, STM32}
\end{abstract}

\end{document}
