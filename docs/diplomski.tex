\documentclass[times, utf8, diplomski]{fer}
\usepackage{booktabs}

\begin{document}

\thesisnumber{3155}

\title{Programska podrška za pouzdano prikupljanje podataka s ugradbenih sustava}

\author{Branimir Ričko}

\maketitle

\zahvala{Zahvaljujem se Lani na tome sto mi je napisala cijeli zavrsni. <3}

\tableofcontents

\chapter{Uvod}
Ugradbena računala nude brojne mogućnosti poboljšanja svakodnevice pojedinaca. Njihove fizičke dimenzije i niska potrošnja cine ih dobrim kandidatom za stvaranjem uređaja i strojeva koji su pametniji on onih proizvedenih u prošlom desetljeću. Iako je bolja budućnost veoma blizu svima nama, iako postoje ugradbena računala koja nam omogućuju izgradnju pametnijih stvari, proces izgradnje nije nimalo trivijalan. Ovaj se rad bavi navedenim procesom i analizom pouzdanosti istog. Opisan je postupak izrade jednostavnog sustava s ugradbenim računalom gdje je ugradbeno računalo stm32, koje čita podatke sa senzora i iste u realnom vremenu \v{s}alje na server.

\chapter{Ugradbena računala}
Ugradbena računala su računala malih dimenzija i niske potro\v{s}nje. 

\section{STM32}
Stm32 je ugradbeno računalo. U ovom radu stm32 koristi se kao primjer računala na kojem je moguće razvijati pouzdan sustav.

\section{Razvojno okru\v{z}enje}
Razvojno okru\v{z}enje je okru\v{z}enje u kojem se razvija.

\subsection{cmake}
Cmake je programski alat koji je koristan za definiranje koraka potrebnih za izgradnju projekta.

\subsection{GNU make}
GNU make je programski alat za definiranje \i{recepta} i kako izgraditi pojedini element sustava.

\begin{equation}
  \pi=3
\end{equation}

\chapter{Zaključak}
Ugradbena računala su dobra.

\bibliography{literatura}
\bibliographystyle{fer}

\begin{sazetak}
  Ugradbena ra\v{c}unala nude brojne mogućnosti poboljšanja svakodnevice pojedinaca. Njihove fizičke dimenzije i niska potrošnja čine ih dobrim kandidatom za stvaranjem uređaja i strojeva koji su pametniji on onih proizvedenih u prošlom desetljeću. Iako je bolja budućnost veoma blizu svima nama, iako postoje ugradbena računala koja nam omogućuju izgradnju pametnijih stvari, proces izgradnje nije nimalo trivijalan. Ovaj se rad bavi navedenim procesom i analizom pouzdanosti istog.

\kljucnerijeci{Ugradbena računala, Pouzdanost, Bluetooth, STM32}
\end{sazetak}

\engtitle{Software support for data collection from embedded devices with high availability}

\begin{abstract}
Embedded computers offer number of features that can augment day to day exisstance of every individual. Their dimensions and power efficiency make them great candidate for making a world a better place. Even thou embedded devices exist and are availabel to everyone, creating a system that uses that device is not trivial. This work touches on those processies and analasys of availability of those processies.

\keywords{Embedded Computers, Availability, Bluetooth, STM32}
\end{abstract}

\end{document}
